% !TEX TS-program = pdflatex
% !TEX encoding = UTF-8 Unicode

% This is a simple template for a LaTeX document using the "article" class.
% See "book", "report", "letter" for other types of document.

\documentclass[11pt]{article} % use larger type; default would be 10pt
\usepackage[italian]{babel}

\usepackage[utf8]{inputenc} % set input encoding (not needed with XeLaTeX)
\usepackage{wrapfig}
%%% Examples of Article customizations
% These packages are optional, depending on whether you want the features they provide.
% See the LaTeX Companion or other references for full information.

%%% PAGE DIMENSIONS
\usepackage{geometry} % to change the page dimensions
\geometry{a4paper} % or letterpaper (US) or a5paper or....
% \geometry{margin=2in} % for example, change the margins to 2 inches all round
% \geometry{landscape} % set up the page for landscape
% read geometry.pdf for detailed page layout information

\usepackage{graphicx} % support the \includegraphics command and options
\usepackage{url} % better links representation, o così dicono

% \usepackage[parfill]{parskip} % Activate to begin paragraphs with an empty line rather than an indent

%%% PACKAGES
\usepackage{amsmath}
\usepackage{booktabs} % for much better looking tables
\usepackage{array} % for better arrays (eg matrices) in maths
\usepackage{paralist} % very flexible & customisable lists (eg. enumerate/itemize, etc.)
\usepackage{verbatim} % adds environment for commenting out blocks of text & for better verbatim
\usepackage{subfig} % make it possible to include more than one captioned figure/table in a single float
% These packages are all incorporated in the memoir class to one degree or another...
\usepackage{listings,multicol}
\usepackage{caption}
\usepackage{comment}

\usepackage{graphicx}
\graphicspath{ {./images/} }

\usepackage{mathpartir}

%%% HEADERS & FOOTERS
\usepackage{fancyhdr} % This should be set AFTER setting up the page geometry
\pagestyle{fancy} % options: empty , plain , fancy
\renewcommand{\headrulewidth}{0pt} % customise the layout...
\lhead{}\chead{}\rhead{}
\lfoot{}\cfoot{\thepage}\rfoot{}

%%% SECTION TITLE APPEARANCE
\usepackage{sectsty}
\allsectionsfont{\sffamily\mdseries\upshape} % (See the fntguide.pdf for font help)
% (This matches ConTeXt defaults)

\usepackage[usenames,dvipsnames]{xcolor}
\usepackage{hyperref} % To refer a link (website)
\hypersetup{%
  colorlinks=true,% hyperlinks will be coloured
  %linkcolor={BrickRed},% hyperlink text will be green
  linkbordercolor=BrickRed,
  citebordercolor=White,
  urlbordercolor=White,
  runbordercolor=White,
  menubordercolor=White,
  filebordercolor=White,
 % urlcolor={BrickRed},
%filecolor={White},
%citecolor={BrickRed},
allcolors={BrickRed}
%allbordercolors={BrickRed}
}

\makeatletter
\Hy@AtBeginDocument{%
  \def\@pdfborder{0 0 1}% Overrides border definition set with colorlinks=true
  \def\@pdfborderstyle{/S/U/W .5}% Overrides border style set with colorlinks=true
                                % Hyperlink border style will be underline of width 1pt
}
\makeatother

%%% ToC (table of contents) APPEARANCE
\usepackage[nottoc,notlof,notlot]{tocbibind} % Put the bibliography in the ToC
\usepackage[titles,subfigure]{tocloft} % Alter the style of the Table of Contents

\renewcommand{\cftsecfont}{\rmfamily\mdseries\upshape}
\renewcommand{\cftsecpagefont}{\rmfamily\mdseries\upshape} % No bold!

\setlength\parindent{0pt} % Set noindent for the whole document
\newcommand{\ES}{\textcolor{red}}


\usepackage{tikz}
\usetikzlibrary{shapes,trees,fit,decorations.pathreplacing,arrows.meta}

\definecolor{codegreen}{rgb}{0,0.6,0}
\definecolor{codegray}{rgb}{0.5,0.5,0.5}
\definecolor{codepurple}{rgb}{0.58,0,0.82}
\definecolor{backcolour}{rgb}{0.95,0.95,0.92}

\lstdefinestyle{mystyle}{
    backgroundcolor=\color{backcolour},   
    commentstyle=\color{codegreen},
    keywordstyle=\color{magenta},
    numberstyle=\tiny\color{codegray},
    stringstyle=\color{codepurple},
    basicstyle=\ttfamily\footnotesize,
    breakatwhitespace=false,         
    breaklines=true,                 
    captionpos=b,                    
    keepspaces=true,                 
    numbers=left,                    
    numbersep=5pt,                  
    showspaces=false,                
    showstringspaces=false,
    showtabs=false,                  
    tabsize=2
}

\lstset{style=mystyle}
%%% END Article customizations

\begin{document}
%\maketitle

%%% Title page
\begin{titlepage}
	\topskip0pt
	%\vspace*{\fill}
	\centering
	\includegraphics[width=\textwidth]{logo.png}\\
	\vspace*{1cm}
	\Large \textsc{Laurea Magistrale in Informatica}
	
	\vspace*{10mm}
	\hrule width \hsize \kern 1mm \hrule width \hsize height 2pt
	\vspace*{5mm}
	\Huge \emph{\textbf{ACMEat}}\\
	\large \emph{\textbf{Relazione del progetto del corso di\\Ingegneria del Software Orientata ai Servizi}}\\
	\vspace*{5mm}
	\hrule width \hsize height 2pt
	\vspace*{1mm}
	\hrule width \hsize \kern 1mm
	
	\vspace*{10mm}
	\begin{minipage}{0.45\textwidth}
		\begin{flushleft} \Large
			\emph{Studenti:}\\
			\Large \textbf{Lorenzo \textsc{BALUGANI}}\\
			\Large \textbf{Alberto \textsc{PAPARELLA}}\\
			\Large \textbf{Mae \textsc{SOSTO}}
		\end{flushleft}
	\end{minipage}	
	\begin{minipage}{0.45\textwidth}
		\begin{flushright} \Large
			\emph{Docenti:}\\
			\Large \textbf{Prof. Ivan \textsc{LANESE}}\\
			\Large \textbf{Prof. Davide \textsc{ROSSI}}
		\end{flushright}
	\end{minipage}
	
	\vspace*{15mm}
	\Large \textsc{Anno Accademico $2021-2022$}
\end{titlepage}

% Table of contents
\addtocontents{toc}{~\hfill\textbf{Page}\par}	% https://texblog.org/2011/09/09/10-ways-to-customize-tocloflot/
\clearpage
\tableofcontents
\thispagestyle{empty}
\addtocontents{toc}{\protect\thispagestyle{empty}}	% http://tex.stackexchange.com/questions/2995/removing-page-number-from-toc
\newpage

\section{Introduzione}

\clearpage

\section{Descrizione del dominio e del problema}
\label{intro}
La società ACMEat propone ai propri clienti un servizio che permette di selezionare un menu da uno fra un insieme di locali convenzionati e farselo recapitare a domicilio.\\
Per poter usufruire del servizio il cliente deve inizialmente selezionare un comune fra quelli nei quali il servizio è attivo. A fronte di questa selezione ACMEat presenta la lista dei locali convenzionati che operano in quel comune e dei menù che offrono. Il cliente può quindi specificare locale e menù di suo interesse e una fascia oraria per la consegna (si tratta di fasce di 15 minuti tra le 12 e le 14 e tra le 19 e le 21).\\
Segue quindi una fase di pagamento che viene gestita attraverso un istituto bancario terzo al quale il cliente viene indirizzato. A fronte del pagamento l’istituto rilascia un token al cliente il quale lo comunica ad ACMEat, che a sua volta lo usa per verificare con la banca che il pagamento sia stato effettivamente completato. A questo punto l’ordine diventa operativo. I clienti possono comunque ancora annullare l’ordine ma non più tardi di un’ora prima rispetto all’orario di consegna. In tal caso ACMEat chiede alla banca l’annullamento del pagamento.\\
ACMEat conosce tutti i locali convenzionati nei vari comuni nei quali opera e i loro giorni e orari di operatività. Nel caso in cui un locale non sia disponibile in un giorno in cui dovrebbe normalmente essere aperto è responsabilità del locale stesso contattare ACMEat entro le 10 del mattino comunicando tale indisponibilità. Entro tale orario vanno anche comunicati cambiamenti dei menu proposti (in mancanza di tale comunicazione si assume che siano disponibili gli stessi del giorno precedente). I locali vengono anche contattati ad ogni ordine per verificare che siano effettivamente in grado di far fronte alla richiesta del cliente. In caso negativo l’accettazione dell’ordine si interrompe prima che si passi alla fase di pagamento.\\
Per la consegna ACMEat si appoggia a più società esterne: per ogni consegna vengono contattate tutte le società che abbiano sede entro 10 chilometri dal comune interessato specificando: indirizzo del locale dove ritirare il pasto, indirizzo del cliente cui recapitarlo e orario previsto di consegna. A fronte di questa richiesta le società devono rispondere entro 15 secondi specificando la loro disponibilità e il prezzo richiesto; ACMEat sceglierà fra le disponibili che avranno risposto nel tempo richiesto quella che propone il prezzo più basso. Nel caso in cui nessuna società di consegna sia disponibile l’ordine viene annullato prima che si passi alla fase di pagamento.

\clearpage

\section{Modellazione delle comunicazioni}

La prima fase di lavoro ha visto la realizzazione di una \emph{coreografia} (riportata in Lst.~\ref{coreo}) con l'obiettivo di modellare le comunicazioni dello scenario descritto nella Sez.~\ref{intro}. Tale coreografia è stata iterativamente raffinata in modo da migliorare il più possibile le sue proprietà di \emph{correctedness}; per motivi di spazio, viene riportata solo l'ultima coreografia frutto di questo lavoro di raffinamento. La coreografia è stata poi proiettata in un \emph{sistema di ruoli}, riportato nella Sez.~\ref{sdr}, in cui vengono distinti i seguenti ruoli: \emph{ACMEat}, la \emph{banca}, il \emph{cliente}, il \emph{fattorino}, il \emph{ristorante} e il \emph{servizio di spedizione}. Infine, viene riportata una modellazione della coreografia anche attraverso un diagramma di coreografia BPMN.

\subsection{Coreografia}

\begin{lstlisting}[extendedchars=true, literate={à}{{\'a}}1 {ì}{{\'i}}1,label=coreo,caption=Coreografia dello scenario di utilizzo di ACMEat]
Coreografia :== (
	CoreografiaRichiestaMenu |
	CoreografiaOrdine |
	ModificaInformazioniLocali
)

CoreografiaRichiestaMenu :== (
	SelezioneComune : Cliente -> ACMEat ;
	InvioListaLocali : ACMEat -> Cliente
)

CoreografiaOrdine :==   (
	Ordine : Cliente -> ACMEat ;
	Richiesta disponibilità ristorante : ACMEat -> Ristorante;
	DisponibilitàR: Ristorante -> ACMEat;
	// Disponibile ?
	(
		// No
		Annullamento procedura : ACMEat -> Cliente
	) + (
		// Sì
		( 
			Richiesta disponibilità SDS : ACMEat -> SDS;
			( //In 15 secondi 
				PreventivoDisponibilità: SDS -> ACMEat +
				1
			)
		)* ;
	// Lista vuota ?
	(
		// Sì
		AnnullamentoC : ACMEat -> Cliente |
		AnnullamentoR : ACMEat -> Ristorante ;
	) + (
		// No
		Contatta SDS costo minore: ACMEat -> SDS ;
		ACK: SDS -> ACMEat;
		CreazioneRichiestaPagamento: ACMEat -> Banca ;
		ConfermaCreazioneTransazione: Banca -> ACMEat ;
		RedirezionePagamento: ACMEat -> Cliente ;
		// Modellazione timeout 
		(
			Pagamento : Cliente -> Banca;
			InvioTokenC : Banca -> Cliente;
			InvioTokenA : Cliente -> ACMEat ;
			RichiestaValidità : ACMEat -> Banca ;
			ValiditàToken : Banca -> ACMEat ;
			// Token valido ?
			(
				// No
				ErrorePagamento: ACMEat -> Cliente   |
				AnnullamentoR : ACMEat -> Ristorante |
				AnnullamentoS : ACMEat -> SDS        |
			) + (
				// Sì
				AttivazioneOrdineR : ACMEat -> Ristorante |
				AttivazioneOrdineS : ACMEat -> SDS        |
				ConfermaOrdine: ACMEat -> Cliente ;

				// Manca meno di un'ora ?
				(
					// No
					// Annullare ?
					(
						// Sì
						AnnullamentoOrdine: Cliente -> ACMEat ;
						(
							AnnullamentoPagamento: ACMEat -> Banca;
							Rimborso: Banca -> Cliente;
						) |
						(
							AnnullamentoR : ACMEat -> Ristorante;
							RicevutoAnnullamento: Ristorante -> ACMEat;
						) |
						(
							AnnullamentoS : ACMEat -> SDS;
							RicevutoAnnullamento: SDS -> ACMEat;
						)
					) + (
						// No
						1
					)
				) + (
					// Sì
					(
						PagamentoR : ACMEat -> Banca;
						RicezionePagamentoR : Banca -> Ristorante;
					) |
					(
						PagamentoS : ACMEat -> Banca;
						RicezionePagamentoS : Banca -> SDS ;
					) |
					(
						ConsegnaMerceF : Ristorante -> Fattorino ;
						ConcegnaMerceC : Fattorino -> Cliente;
						ConfermaRicevutaSpedizione: Fattorino -> ACMEat;
					)
				)
			) + (
				//Scadenza timer
				ErrorePagamento: ACMEat -> Cliente |
				(
					Annullamento: ACMEat -> Ristorante;
					RicevutoAnnullamento: Ristorante -> ACMEat;    
				) |
				(
					Annullamento: ACMEat -> Corriere ;
					RicevutoAnnullamento: Corriere -> ACMEat;
				)
			)
		)
	)
)

ModificaInformazioniLocali :==  (
	RichiestaAggiornamento : Ristorante -> ACMEat ;
	// Prima delle 10 ?
	(
		// No
		RichiestaRifiutata : ACMEat -> Ristorante
	) + (
		// Sì
		RichiestaAccettata : ACMEat -> Ristorante
	)
)
\end{lstlisting}

\subsection{Correctedness}

\ldots

\clearpage

\subsection{Sistema di Ruoli}
\label{sdr}

\subsubsection{ACMEat}

\begin{lstlisting}[extendedchars=true, literate={à}{{\'a}}1 {ì}{{\'i}}1,caption=Proiezione della coreografia relativamente ad ACMEat]
Coreografia :== (
	CoreografiaRichiestaMenu |
	CoreografiaOrdine |
	ModificaInformazioniLocali 
)  

CoreografiaRichiestaMenu :==  (
	SelezioneComune@Cliente; 
	InvioListaLocali@Cliente 
) 

CoreografiaOrdine :==   ( 
	Ordine@Cliente ;
	Richiesta disponibilità ristorante@Ristorante ;
	DisponibilitàR@Ristorante ;
	// Disponibile ?
	(
		// No
		Annullamento procedura@Cliente
	) + (
		// Sì
		(
			Richiesta disponibilità SDS@SDS ; 
			(
				// In 15 secondi
				PreventivoDisponibilità@SDS +
				1
			)
	)* ; 
	// Lista vuota ? 
	(
		// Sì
		AnnullamentoC@Cliente |
		AnnullamentoR@Ristorante ;
	) + (
		// No
		Contatta SDS costo minore@SDS ;
		ACK@SDS ;
		CreazioneRichiestaPagamento@Banca ;
		ConfermaCreazioneTransazione@Banca ;
		RedirezionePagamento@Cliente ;
		// Modellazione timeout 
		(
			1;
			1;
			InvioTokenA@Cliente ;
			RichiestaValidità@Banca ;
			ValiditàToken@Banca ;
			// Token valido ?
			(
				// No
				ErrorePagamento@Cliente |
				AnnullamentoR@Ristorante |
				AnnullamentoS@SDS |
			) + (
				// Sì
				AttivazioneOrdineR@Ristorante |
				AttivazioneOrdineS@SDS |
				ConfermaOrdine@Cliente ;
				// Manca meno di un'ora ?
				(
					// No
					// Annullare ?
					(
						// Sì
						AnnullamentoOrdine@Cliente;
						(
							AnnullamentoPagamento@Banca ;
							1 ;
						) |
						(
							AnnullamentoR@Ristorante ;
							RicevutoAnnullamento@RistorantE ;
						) |
						(
							AnnullamentoS@SDS ;
							RicevutoAnnullamento@SDS ;
						)
					) +
					(
						// No
						1
					) 
				) +
				(
					// Sì
					(
						PagamentoR@Banca ;
						1 ;
					) |
					(
						PagamentoS@Banca ;
						1 ;
					) |
					(
						1 ;
						1 ;
						ConfermaRicevutaSpedizione@Fattorino ;
					)
				)
			) +
			(
				//Scadenza timer
				ErrorePagamento@Cliente |
				(
					Annullamento@Ristorante ;
					RicevutoAnnullamento@Ristorante ;
				) |
				(
					Annullamento@Corriere ;
					RicevutoAnnullamento@Corriere ;
				)
			)
		)
	)
) 

ModificaInformazioniLocali :==  (
	RichiestaAggiornamento@Ristorante ;
	// Prima delle 10 ?
	(
		// No
		RichiestaRifiutata@Ristorante
	) +
	(
		// Sì
		RichiestaAccettata@Ristorante
	) 
) 
\end{lstlisting}

\clearpage

\subsubsection{Banca}

\begin{lstlisting}[extendedchars=true, literate={à}{{\'a}}1 {ì}{{\'i}}1,caption=Proiezione della coerografia relativamente alla banca]
Coreografia :== (
	CoreografiaRichiestaMenu |
	CoreografiaOrdine |
	ModificaInformazioniLocali
) 

CoreografiaRichiestaMenu :== (
	1 ;
	1
) 

CoreografiaOrdine :== (
	1 ;
	1 ;
	1 ;
	// Disponibile ?
	(
		// No
		1
	) +
	(
		// Sì
		( 
			1 ;
			(
				 // In 15 secondi
				1 +
				1
			)
		)* ;
		// Lista vuota ?
		(
			// Sì
			1 |
			1 ;
		) +
		(
			// No
			1 ;
			1 ;
			CreazioneRichiestaPagamento@ACMEat ;
			ConfermaCreazioneTransazione@ACMEat ;
			1 ;
			// Modellazione timeout
			(
				Pagamento@Cliente ;
				InvioTokenC@Cliente ;
				1 ;
				RichiestaValidità@ACMEat ;
				ValiditàToken@ACMEat ;
				// Token valido ?
				(
					// No
					1 |
					1 |
					1 |
				) +
				(
					// Sì
					1 |
					1 |
					1 ;
					// Manca meno di un'ora ?
					(
						// No
						// Annullare ?
						(
							// Sì
							1 ;
							(
								AnnullamentoPagamento@ACMEat ;
								Rimborso@Cliente ;
							) |
						(
							1 ;
							1 ;
						) |
						(
							1 ;
							1 ;
						)
					) +
					(
						// No
						1
					)
				) +
				(
					// Sì
					(
						PagamentoR@ACMEat ;
						RicezionePagamentoR@Ristorante ;
					) |
					(
						PagamentoS@ACMEat ;
						RicezionePagamentoS@SDS ;
					) |
					(
						1 ;
						1 ;
						1 ;
					)
				)
			) +
			(
				// Scadenza timer
				1 |
				(
					1 ;
					1 ;
				) |
				(
					1 ;
					1 ;
				)
			)
		)
	)
)

ModificaInformazioniLocali :== (
	1 ;
	// Prima delle 10 ?
	(
		// No
		1
	) +
	(
		// Sì
		1
	)
)
\end{lstlisting}

\clearpage

\subsubsection{Cliente}

\begin{lstlisting}[extendedchars=true, literate={à}{{\'a}}1 {ì}{{\'i}}1, caption=Proiezione della coreografia relativamente al cliente]
Coreografia :== (
	CoreografiaRichiestaMenu |
	CoreografiaOrdine |
	ModificaInformazioniLocali
) 

CoreografiaRichiestaMenu :== (
	1 ;
	InvioListaLocali@ACMEat
) 

CoreografiaOrdine :== (
	Ordine@ACMEat ;
	1 ;
	1 ;
	// Disponibile ?
	(
		// No
		Annullamento procedura@ACMEat
	) +
	(
		// Sì
		(
			1 ;
			(
				// In 15 secondi
				1 +
				1
			)
		)* ;
		// Lista vuota ?
		(
			// Sì
			AnnullamentoC@ACMEat |
			1;
		) +
		(
			// No
			1 ;
			1 ; 
			1 ;
			1 ;
			RedirezionePagamento@ACMEat ;
			// Modellazione timeout
			(
				Pagamento@Banca ;
				InvioTokenC@Banca ;
				InvioTokenA@ACMEat ;
				1 ;
				1 ;
				// Token valido ?
				(
					// No
					ErrorePagamento@ACMEat |
					1 |
					1
				) +
				(
					// Sì
					1 |
					1 |
					ConfermaOrdine@ACMEat ;
					// Manca meno di un'ora ?
					(
						// No
						// Annullare ?
						(
							// Sì
							AnnullamentoOrdine@ACMEat ;
							(
								1 ;
								Rimborso@Banca ;
							) |
							(
								1 ;
								1 ;
							) |
							(
								1 ;
								1
							)
						) +
						(
							// No
							1
						)
					) +
					(
						// Sì
						(
							1 ;
							1
						) |
						(
							1 ;
							1
						) |
						(
							1 ;
							ConcegnaMerceC@Fattorino ;
							1
						)
					)
				) +
				(
					//Scadenza timer
					ErrorePagamento@ACMEat |
					(
						1 ;
						1
					) |
					(
						1 ;
						1
					)
				)
			)
		)
	)
)

 

ModificaInformazioniLocali :== (
	1 ;
	// Prima delle 10 ?
	(
		// No
		1
	) +
	(
		// Sì
		1
	)
)
\end{lstlisting}

\clearpage

\subsubsection{Fattorino}

\begin{lstlisting}[extendedchars=true, literate={à}{{\'a}}1 {ì}{{\'i}}1, caption=Proiezione della coreografia relativamente al fattorino]
Coreografia :== (
	CoreografiaRichiestaMenu |
	CoreografiaOrdine |
	ModificaInformazioniLocali
)

CoreografiaRichiestaMenu :== (
	1 ;
	1
) 

CoreografiaOrdine :== (
	1 ;
	1 ;
	1 ;
	// Disponibile ?
	(
		// No
		1
	) +
	(
		// Sì
		(
			1 ;
			( 
				// In 15 secondi
				1 +
				1
			)
		)* ;
		// Lista vuota ?
		(
			// Sì
			1 |
			1
		) +
		(
			// No
			1 ;
			1 ;
			1 ;
			1 ;
			1 ;
			// Modellazione timeout 
			(
				1 ;
				1 ;
				1 ;
				1 ;
				1 ;
				// Token valido ?
				(
					// No
					1 |
					1 |
					1
				) +
				(
					// Sì
					1 |
					1 |
					1 ;
                                        	// Manca meno di un'ora ?
					(
						// No
						// Annullare ?
						(
							// Sì
							1 ;
							(
								1 ;
								1
							) |
							(
								1 ;
								1
							) |
							(
								1 ;
								1
							)
						) +
						(
							// No
							1
						)
					) +
					(
						// Sì
						(
							1 ;
							1
						) |
						(
							1 ;
							1
						) |
						(
							ConsegnaMerceF@Ristorante ;
							ConcegnaMerceC@Cliente ;
							ConfermaRicevutaSpedizione@ACMEat
						)
					)
				) +
				(
					// Scadenza timer
					1 |
					(
						1 ;
						1
					) |
					(
						Annullamento@ACMEat ;
						RicevutoAnnullamento@ACMEat ;
					)
				)
			)
		)
	)
)

ModificaInformazioniLocali :== (
	1 ;
	// Prima delle 10 ?
	(
		// No
		1
	) + (
		// Sì
		1
	)
)
\end{lstlisting}

\clearpage

\subsubsection{Ristorante}

\begin{lstlisting}[extendedchars=true, literate={à}{{\'a}}1 {ì}{{\'i}}1, caption=Proiezione della coreografia relativamente al ristorante]
Coreografia :== (
	CoreografiaRichiestaMenu |
	CoreografiaOrdine |
	ModificaInformazioniLocali
) 

CoreografiaRichiestaMenu :== (
	1 ;
	1
)

CoreografiaOrdine :== (
	1 ;
	Richiesta disponibilità ristorante@ACMEat ;
	DisponibilitàR@ACMEat ;
	// Disponibile ?
	(
		// No
		1
	) +
	(
		// Sì
		(
			1 ;
			(
				// In 15 secondi
				1 +
				1
			)
		)* ;
		// Lista vuota ?
		(
			// Sì
			1 |
			AnnullamentoR@ACMEat
		) +
		(
			// No
			1 ;
			1 ;
			1 ;
			1 ;
			1 ;
			// Modellazione timeout
			(
				1 ;
				1 ;
				1 ;
				1 ;
				1 ;
				// Token valido ?
				(
					// No
					1 |
					AnnullamentoR@ACMEat |
					1
				) +
				(
					// Sì
					AttivazioneOrdineR@ACMEat |
					1 |
					1 ;
					// Manca meno di un'ora ?
					(
						// No
						// Annullare ?
						(
							// Sì
							1 ;
							(
								1 ;
								1
							) |
							(
								AnnullamentoR @ACMEat ;
								RicevutoAnnullamento@ACMEat
							) |
							(
								1 ;
								1
							)
						) +
						(
							// No
							1
						)
					) +
					(
						// Sì
						(
							1 ;
							RicezionePagamentoR@Banca ;
						) |
						(
							1 ;
							1
						) |
						(
							1 ;
							1;
							1
						)
					)
				) +
				(
					// Scadenza timer
					1 |
					(
						Annullamento@ACMEat ;
						RicevutoAnnullamento@ACMEat
					) |
					(
						1 ;
						1
					)
				)
			)
		)
	)
)
 

ModificaInformazioniLocali :== (
	RichiestaAggiornamento@ACMEat ;
	// Prima delle 10 ?
	(
		// No
		RichiestaRifiutata@ACMEat
	) +
	(
		// Sì
		RichiestaAccettata@ACMEat
	)
) 
\end{lstlisting}

\clearpage

\subsubsection{Servizio di spedizione}

\begin{lstlisting}[extendedchars=true, literate={à}{{\'a}}1 {ì}{{\'i}}1, caption=Proiezione della coreografia relativamente al servizio di spedizione]
Coreografia :== (
	CoreografiaRichiestaMenu |
	CoreografiaOrdine |
	ModificaInformazioniLocali
)

CoreografiaRichiestaMenu :== (
	1 ;
	1
)

CoreografiaOrdine :==   (
	1 ;
	1 ;
	1 ;
	// Disponibile ?
	(
		// No
		1
	) +
	(
		// Sì
		(
			Richiesta disponibilità SDS@ACMEat ;
			(
				// In 15 secondi
				PreventivoDisponibilità@ACMEat +
				1
			)
		)* ;
		// Lista vuota ?
		(
			// Sì
			1 |
			1
		) +
		(
			// No
			Contatta SDS costo minore@ACMEat ;
			ACK@ACMEat ;
			1 ;
			1 ;
			1 ;
			// Modellazione timeout
			(
				1 ;
				1 ;
				1 ;
				1 ;
				1 ;
				// Token valido ?
				(
					// No
					1 |
					1 |
					AnnullamentoS@ACMEat |
				) +
				(
					// Sì
					1 |
					AttivazioneOrdineS@ACMEat |
					1 ;
					// Manca meno di un'ora ?
					(
						// No
						// Annullare ?
						(
							// Sì
							1 ;
							(
								1 ;
								1
							) |
							(
								1 ;
								1
							) |
							(
								AnnullamentoS@ACMEat ;
								RicevutoAnnullamento@ACMEat
							)
						) +
						(
							// No
							1
						)
					) +
					(
						// Sì
						(
							1 ;
							1
						) |
						(
							PagamentoS : ACMEat -> Banca ;
							RicezionePagamentoS : Banca -> SDS ;
					) |
					(
						1 ;
						1 ;
						1
					)
				)
			) +
			(
				// Scadenza timer
				1 |
				(
					1 ;
					1
				) |
				(
					1 ;
					1
				)
			)
		)
	)
) 

ModificaInformazioniLocali :== (
	1 ;
	// Prima delle 10 ?
	(
		// No
		1
	) +
	(
		// Sì
		1
	)
) 
\end{lstlisting}

\clearpage

\section{Documentazione}

Durante la seconda fase di lavoro è stato realizzanto un diagramma di collaborazione BPMN (Fig.~\ref{bpmn}) con l'obiettivo di modellare l'intera realtà descritta a scopo documentativo, compresi i dettagli di ogni partecipante. Tale diagramma e la relativa export \verb|.png| sono forniti in allegato a questa relazione.

\begin{figure}[!ht]
\includegraphics[width=\textwidth]{bpmn}
\caption{Diagramma di collaborazione BPMN}
\label{bpmn}
\end{figure}

Di seguito, riportiamo alcuni estratti rilevanti di questo diagramma.

\begin{figure}[!ht]
\begin{center}
\fbox{\includegraphics[scale=0.5]{bpmn1}}
\caption{Scambio di messaggi iniziale fra un cliente e ACMEat}
\end{center}
\end{figure}

\begin{figure}[!ht]
\begin{center}
\fbox{\includegraphics[scale=0.5]{bpmn2}}
\vspace*{0.5cm}\\
\fbox{\includegraphics[scale=0.5]{bpmn3}}
\caption{Scambio di messaggi fra ACMEat e un ristorante per verificarne la disponibilità}
\end{center}
\end{figure}

\begin{figure}[!ht]
\begin{center}
\fbox{\includegraphics[scale=0.5]{bpmn4}}
\vspace*{0.5cm}\\
\fbox{\includegraphics[scale=0.5]{bpmn5}}
\caption{Scelta del servizio di spedizione all'interno di ACMEat}
\end{center}
\end{figure}

\begin{figure}[!ht]
\begin{center}
\fbox{\includegraphics[width=\textwidth]{bpmn6}}
\caption{Gestione pagamento dopo redirezione cliente da parte di ACMEat}
\end{center}
\end{figure}

\begin{figure}[!ht]
\begin{center}
\fbox{\includegraphics[width=\textwidth]{bpmn7}}
\caption{Gestione (annullamento) ordine da parte di ACMEat }
\end{center}
\end{figure}

\begin{figure}[!ht]
\begin{center}
\fbox{\includegraphics[width=\textwidth]{bpmn8}}
\fbox{\includegraphics[scale=0.5]{bpmn9}}
\caption{Gestione richiesta aggiornamento informazioni ristorante da parte di ACMEat (sopra) e del ristorante (sotto)}
\end{center}
\end{figure}

\begin{figure}[!ht]
\begin{center}
\fbox{\includegraphics[scale=0.5]{bpmn10}}
\caption{Gestione pagamento da parte della banca}
\end{center}
\end{figure}

\begin{figure}[!ht]
\begin{center}
\fbox{\includegraphics[scale=0.5]{bpmn11}}
\caption{Verifica token da parte di ACMEat}
\end{center}
\end{figure}

\begin{figure}[!ht]
\begin{center}
\fbox{\includegraphics[width=\textwidth]{bpmn12}}
\caption{Gestione rimborsi e pagamenti da parte della banca}
\end{center}
\end{figure}

\begin{figure}[!ht]
\begin{center}
\fbox{\includegraphics[scale=0.5]{bpmn13}}
\caption{Notifica disponibilità servizio di spedizione}
\end{center}
\end{figure}

\begin{figure}[!ht]
\begin{center}
\fbox{\includegraphics[width=\textwidth]{bpmn14}}
\caption{Gestione consegna da parte di servizio di spedizione e fattorino}
\end{center}
\end{figure}

\clearpage

\section{Progettazione}

La terza fase di lavoro ha visto la progettazione di una SOA per la realizzazione del sistema, documentata utilizzando UML.

\clearpage

\section{Sviluppo}

La quarta fase di lavoro ha visto la realizzazione del sistema. Sono state usate come tecnologie un BPMS (Camunda), Jolie e API Rest. In particolare:
\begin{itemize}
\item il servizio ACMEat rende accessibili capabilities realizzate attraverso il BPMS;
\item i servizi esterni ad ACMEat sono GIS, servizi bancari, locali e società di consegna;
\item i servizi bancari sono realizzati in Jolie;
\item GIS, i locali e le società di consegna sono realizzati con API Rest.
\item il dialogo fra Jolie e BPMS avviene via SOAP
\end{itemize}

\clearpage

\section{Conclusioni}

\end{document}